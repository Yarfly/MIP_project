\documentclass{article}
\usepackage{graphicx} % Required for inserting images

\title{Yaroslav Melnyk rslvmelnyk@gmail.com 17.10.2024}
\author{Yaroslav Melnyk}
\date{November 2024}

\begin{document}

\maketitle

\begin{abstract}
    For a large organization, personalization is a key factor of
involvement and satisfaction. YouTube Music's personalization
strategy uses various technologies to provide high-quality music.
The process of content filtering and filtering together provides the
best recommendations and analysis of search behavior. Aggregate
filtering analyzes people's interaction with music, including song
title, time and mood.
Deep learning models can analyze large amounts of data to
understand the relationship between user preferences and
behavior. The ranking algorithm determines which songs are
selected first, so the most popular songs move up the rating. In
addition, data can provide better results at any time depending on
factors such as the time of day or the user's location.
New methods, such as consistent recommendations, have been
developed to improve efficiency and decision-making processes.
The matrix output method helps to process large data sets and
describes the relationship between variables and distributions.
By providing this tool, YouTube Music ensures that its
recommendation system will be adapted to the needs of each user,
making the experience more personalized and enjoyable.
\end{abstract}

\end{document}
